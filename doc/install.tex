% This file is part of the imaging2 class library.
%
% University of Innsbruck, Infmath Imaging, 2009.
% http://infmath.uibk.ac.at
%
% All rights reserved.


\documentclass[a4paper,10pt]{article}
\usepackage{hyperref}
\usepackage{verbatim}

\title{ {\bf{HowTo}} \\
        Install the imaging2 library on a Unix system \\
        {\small{developed by Matthias Fuchs et al.}}
      }
\author{\small{Thomas Fidler}}

\begin{document}
\maketitle

\section{Agreements}
This installation guide utilises some programs which are neccessary for the installation process. They can be replaced (except cmake)
by any suitable equivalent program the user may be more familiar with.
\begin{itemize}
    \item   xterm: a simple terminal window; Start: Alt+F2 -[command] xterm \&
    \item   cmake: a build system, Start: Alt+F2 -[command] ccmake .
    \item   kwrite: a text editor; Start: Alt+F2 -[command] kwrite \&
\end{itemize}
Furthermore, we use some notation:
\begin{tabbing}
    home\_folderblaa    \=  blabla \kill
    home\_folder        \>  home folder of your linux account \\
    {[shell]}           \>  shell prompt of your system; \\
                        \>  depending on your system another string than \emph{shell} appears \\
    {[cmake]}           \>  indicates the user interface of cmake \\
    {[kwrite]}          \>  indicates the user interface of kwrite \\
\end{tabbing}
To get the information about your \emph{home\_folder} open a terminal window and type
\begin{verbatim}
    [shell] cd ~
    [shell] pwd
\end{verbatim}
Folders and files names are written in italics in text paragraphs.

Moreover we assume that the following libraries and their header files are installed:
\begin{itemize}
 \item Lapack and BLAS
 \item Boost
 \item LibXML2
 \item LibPNG
 \item ImageMagick and ImageMagick++
 \item OpenGL and GLUT
 \item a Fortran compiler (e.g. GCC Fortran)
\end{itemize}
These libraries are available as packages for most Linux-distributions. 
Make sure that you install the library packages and the corresponding developer packages (which include the header files).

Note that it is possible to locally install the imaging2 library on your cluster account.
You simply have to login on the cluster with the following command:
\begin{verbatim}
    [shell] ssh -X username@fw02-c703.uibk.ac.at
\end{verbatim}
Then follow the instructions below.
 

\section{Installation Guide for Library \emph{Spooles}}
First download the modified compressed package from the \emph{Infmath} server\footnote{\url{http://infmath.uibk.ac.at/~matthiasf/spooles.tar.gz}}
and save it local to \emph{home\_folder}. Next, open a terminal window and type in the
following commands:
\begin{verbatim}
    [shell] cd
    [shell] ls
    [shell] tar -xzf spooles.tar.gz
    [shell] cd spooles
    [shell] make lib
\end{verbatim}
In the directory spooles we have to rename the file \emph{spooles.a}:
\begin{verbatim}
    [shell] mv spooles.a libspooles.a
\end{verbatim}

\section{Installation Guide for Library \emph{imaging2}}
In the next step the files of the imaging2 library have to be downloaded via svn. For that purpose open a new terminal window
and be sure to be in your \emph{home\_folder}.
\begin{verbatim}
    [shell] svn checkout http://infmath.uibk.ac.at:8080/
            share/code/imaging2/trunk/ imaging2
\end{verbatim}
You may be prompted to specify your password.
Press \emph{Enter} once (without entering a password).
You will now be asked to provide your user name followed by your password (cf. internal Wiki).
Afterwards change to the subfolder \emph{imaging2} and start cmake:
\begin{verbatim}
    [shell] cd imaging2
    [shell] ccmake .
\end{verbatim}
Now the user interface of cmake appears: just press
\begin{verbatim}
    [cmake] press c
    [cmake] change by arrow down to CMAKE_BUILD_TYPE
    [cmake] press return
    [cmake] type "Debug"
    [cmake] press return
    [cmake] press c
    [cmake] press g
    [cmake] press t
    [cmake] change by arrow down to CMAKE_C_FLAGS
    [cmake] press return
    [cmake] type "-I/home_folder/spooles"
    [cmake] press return
    [cmake] change by arrow down to CMAKE_EXE_LINKER_FLAGS
    [cmake] press return
    [cmake] type "-L/home_folder/spooles"
    [cmake] press return
    [cmake] press c
    [cmake] press g
\end{verbatim}
Now, hope everything works well:
\begin{verbatim}
    [shell] make
\end{verbatim}
To verify the compilation process we check the following:
\begin{verbatim}
    [shell] cd core/ublas/
    [shell] ./fixed_ublas
\end{verbatim}
A list of numbers and vectors should be displayed in the terminal.
\begin{verbatim}
    [shell] cd ..
    [shell] cd ..
    [shell] cd fem
    [shell] ./fem
\end{verbatim}
The Lenna image and a black square on white background are filtered using diffusion flow, mean curvature flow and total variation flow.
\begin{verbatim}
    [shell] cd ..
    [shell] cd polytope
    [shell] ./polytope polytope.xml
\end{verbatim}
Four intersecting polytopes should be displayed in a window; the union of the polytopes should be coloured.
\begin{verbatim}
    [shell] cd ..
    [shell] cd shape/mrep/
    [shell] ./mrep mrep.xml
\end{verbatim}
A window with two shapes and their corresponding M-Rep model should appear.

\section{Creating Test File}
In the following we check if the installation process was successful by creating a simple test program which
includes the library imaging2. First create a new subfolder (i.e. \emph{/home\_folder/seminar}) and open an
editor (i.e. kedit).
\begin{verbatim}
    [shell] cd ~
    [shell] mkdir seminar
    [shell] cd seminar
    [shell] kedit &
\end{verbatim}
Type in the following code fragment and save it, i.e. as \emph{seminar.cpp} in \emph{/home\_folder/seminar}.
\begin{verbatim}
    [kwrite] #include<core/imaging2.hpp>

             int main(int argc, char * const argv[])
             {
                 return 0;
             }
\end{verbatim}
Open another kedit session, type in the following code fragment and save it as \emph{CMakeLists.txt} in
\emph{/home\_folder/seminar}.
\begin{verbatim}
    [kwrite] project(seminar CXX)

             add_executable(seminar seminar.cpp)
             target_link_libraries(seminar
                 imaging2)
\end{verbatim}
Try to compile the file \emph{seminar.cpp} with the aid of cmake and \emph{CMakeLists.txt}.
\begin{verbatim}
    [shell] ccmake .
    [cmake] press c
    [cmake] change by arrow down to CMAKE_BUILD_TYPE
    [cmake] press return
    [cmake] type "Debug"
    [cmake] press return
    [cmake] press c
    [cmake] press t
    [cmake] change by arrow down to CMAKE_CXX_FLAGS
    [cmake] press return
    [cmake] type "-I/home_folder/imaging2"
    [cmake] press return
    [cmake] change by arrow down to CMAKE_EXE_LINKER_FLAGS
    [cmake] press return
    [cmake] type "-L/home_folder/imaging2"
    [cmake] press return
    [cmake] press c
    [cmake] press g
\end{verbatim}
After telling the compiler where to find the header file \emph{imaging2.hpp} the program should compile successfully and an executable \emph{seminar} will be generated.
\begin{verbatim}
    [shell] make
    [shell] ./seminar
\end{verbatim}
\end{document}
